% Tên đề tài: 'NHẬN DẠNG ẢNH VÀ MỘT SỐ ỨNG DỤNG TRONG Y HỌC
% =========================================================================
% Chủ nhiệm đề tài:			Trần Nam Hưng
% Thành viên:				Lý Ngọc Thanh
% GitHub repository:	https://github.com/hungtrannam/image_classification
% =========================================================================

\documentclass[a4paper,oneside]{report}			% Viết trên giấy A4 và in một mặt
% ====================================================================
% Về font chữ, canh lề, size chữ các tiêu đề
\usepackage[english]{babel}
\usepackage[utf8]{vietnam}												% Gõ Tiếng Việt trong LaTex
\usepackage{scrextend}
\changefontsizes{13pt}													% Chỉnh size chữ	
\usepackage{titlesec}													% Chỉnh size, font của các tiêu đề
\usepackage{times}													% Chỉnh font chữ thành Times
\usepackage{mathptm}													% Chỉnh font các ký hiệu toán thành Times
\usepackage[top=2cm,inner=3cm,outer=2cm,bottom=2cm,headheight=2cm]{geometry}						% Dùng để tùy chỉnh lề trái phải
\setlength{\parindent}{1cm}												% Tùy chỉnh thụt đầu dòng mỗi đoạn
\renewcommand{\baselinestretch}{1.3}											% Lệnh tùy chỉnh dãn cách các dòng
\usepackage{amsmath,amssymb,yhmath,mathrsfs,fontawesome}								% Dùng để chèn các công thức Toán
\usepackage{amsfonts}													% Dùng để chèn các công thức Toán
\usepackage[standard,thmmarks]{ntheorem}										%thref,amsmath,hyperref
%\usepackage{indentfirst}												% Thụt đầu đòng
\usepackage[margin=11pt,font=small, labelfont=bf]{caption} 								% Tùy chỉnh font, size của caption

% Về các bảng, hình, màu sắc và liên kết
\usepackage{fancybox} 
\usepackage{array} 													% Dùng để kẻ các đường trong bảng
\usepackage{booktabs}													% Dùng để kẻ các đường trong bảng
\usepackage{multirow}													% Dùng để chia các hàng trong các bảng
\usepackage{makecell}													% Dùng để chia các cột trong bảng
\usepackage{xtab}

\usepackage[bookmarksnumbered, hyperindex, unicode]{hyperref}
\usepackage{graphicx}													% Dùng để chèn hình ảnh
\graphicspath{ {images/} }												% Folder chứa hình ảnh
\usepackage{subfiles}													% Dùng để chia thành các files nhỏ
\usepackage{fancyhdr}													% Dùng để tạo header và footer
\usepackage{emptypage}													% Không đánh số trang, header, footer với các trang trắng
\usepackage{tabularx}													% Dùng để chèn bảng
\usepackage{pdflscape}													% Một vài trang nằm ngang (landscape) thay vì dọc (portrait) (bảng, biểu đồ)
\usepackage[square]{natbib}												% Dùng cho tài liệu tham khảo
\setcitestyle{super}													% Ký hiệu tài liệu tham khảo dạng superscript
\usepackage{url}													% Dùng cho URL
\usepackage{longtable}													% Dùng cho các bảng dài, nhiều trang
\usepackage{tikz}													% Dùng để vẽ hình, vẽ bảng biến thiên
\usetikzlibrary{calc}
\usepackage{floatrow}
\floatsetup[table]{capposition=top}
\usepackage{enumerate}													% Dùng để làm nổi bật header của bảng
\usepackage{float}													% Dùng để thiết đặt thuộc tính H (here) đặt tại chỗ cho ảnh

% Định dạng cho header của bảng
\renewcommand\theadfont{\bfseries}
\usepackage{listings}													% Dùng để viết các mã code trong Matlab và các chương trình khác
\usepackage{color} 													%red, green, blue, yellow, cyan, magenta, black, white
\usepackage{pdfpages}													% Chèn các file .pdf vào LaTex
\usepackage[ruled,lined,commentsnumbered, norelsize]{algorithm2e}							% Dùng để viết các thuật toán
\usepackage{makeidx}													%Tạo index
\usepackage{imakeidx}
\usepackage{qrcode}													% Tạo mã QR
\makeindex[title = INDEX,intoc]


% ===================Tiêu đề và tác giả===============================
\title{Nhận dạng ảnh và ứng dụng trong y học}
\author{Trần Nam Hưng}	
% ====================================================================

% Header và footer 
\pagestyle{fancy}
\fancyhf{}
\renewcommand{\headrulewidth}{0pt}	% Tùy chỉnh độ dày của thanh ngang header
\cfoot{\fontsize{11}{11} \selectfont \thepage}

%=================================================================
%			CÀI ĐẶT TÀI LIỆU
%=================================================================

% ==========Lệnh tùy chỉnh kích cỡ chữ các tiêu đề===================
\titleformat*{\section}{\Large\bfseries}
\titleformat*{\subsection}{\large\bfseries}
\titleformat{\chapter}[display]
{\normalfont\LARGE\bfseries}{\chaptertitlename\ \thechapter}{20pt}{\LARGE}



%===========Lệnh cài đặt màu=========================================
\definecolor{mygreen}{RGB}{28,172,0}
\definecolor{mylilas}{RGB}{170,55,241}

%===========Lệnh cài đặt viết mã code Matlab trong Late==============
\lstset{
	language=Matlab,%
	%basicstyle=\color{red},
	breaklines=true,%
	morekeywords={matlab2tikz},
	keywordstyle=\color{blue},%
	morekeywords=[2]{1}, keywordstyle=[2]{\color{black}},
	identifierstyle=\color{black},%
	stringstyle=\color{mylilas},
	commentstyle=\color{mygreen},%
	showstringspaces=false,%without this there will be a symbol in the places where there is a space
	numbers=left,%
	numberstyle={\tiny \color{black}},% size of the numbers
	numbersep=9pt, % this defines how far the numbers are from the text
	emph=[1]{for,end,break},emphstyle=[1]\color{red}, %some words to emphasise
	%emph=[2]{word1,word2}, emphstyle=[2]{style},  
	basicstyle=\footnotesize %\ttfamily
}

%===========Lệnh cài đặt môi trường định lý====================
\theoremstyle{plain}
\theoremheaderfont{\normalfont\bfseries}
\theorembodyfont{\slshape}
\theoremseparator {.} 
\renewtheorem{theorem}{\hspace*{0.5cm}Định lý} [chapter]
\newcommand{\dly}[1]{\begin{theorem}#1\end{theorem}}

%===========Lệnh cài đặt môi trường định nghĩa====================
\theoremstyle{plain}
\theoremheaderfont{\bfseries}
\theorembodyfont{\normalfont} 
\theoremseparator {.}         
\renewtheorem{definition}{\hspace*{0.5cm}Định nghĩa}[chapter] 
\newcommand{\dng}[1]{\begin{definition}#1\end{definition}}

%===========Lệnh cài đặt môi trường chứng minh====================
\theoremstyle{nonumberplain} 
\theoremheaderfont{\bfseries\slshape}
\theorembodyfont{\normalfont}
\theoremsymbol{\ensuremath{_\blacksquare}}
\renewtheorem{proof}{\hspace*{0.5cm}Chứng minh}
\newcommand{\chm}[1]{\begin{proof}#1\end{proof}}

%===========Lệnh gõ tắt hệ và, hệ hoặc
\newcommand{\hoac}[1]{\left[\begin{aligned}#1\end{aligned}\right.}
\newcommand{\heva}[1]{\left\{\begin{aligned}#1\end{aligned}\right.}

%=================================================================
%			BẮT ĐẦU TÀI LIỆU
%=================================================================

\begin{document}
\pagenumbering{roman}		% Kiểu số trang: i, ii, iii, iv, v,...	


\subfile{bia-ngoai}		% Gọi file bìa ngoài
\subfile{bia-trong1}		% Gọi file bìa trong

\begin{abstract}
	Image classification in machine learning and statistics is the problem of indentifying to which of a set of categories (sub-populations) a new observation belongs, on the basis of a training set of data containing obversation (or instances) whose category membership is known.
	
	One principal thesis in this report 

	\textbf{Keyword:} \textit{Image classification,}
\end{abstract}


\tableofcontents			% Tạo mục lục
\listoffigures				% Tạo danh sách các hình ảnh
\listoftables				% Tạo danh sách các bảng
%\lstlistoflistings			% Tạo danh sách các chương trình máy tính
\listofalgorithms


\chapter*{Danh mục ký hiệu và viết tắt}
\subfile{chapters/viet-tat}
\clearpage

\chapter*{LỜI CẢM ƠN}\addcontentsline{toc}{chapter}{LỜI CẢM ƠN}
\subfile{chapters/loi-cam-on}



\chapter*{PHẦN MỞ ĐẦU}\addcontentsline{toc}{chapter}{PHẦN MỞ ĐẦU}
\subfile{chapters/mo-dau}

	
\chapter{KIẾN THỨC CHUẨN BỊ}
\pagenumbering{arabic}			
\setcounter{page}{1}	% Cài đặt lại số trang

\subfile{chapters/chapter01}	

\chapter[MỘT SỐ PHƯƠNG PHÁP PHÂN LOẠI CHO DỮ LIỆU RỜI RẠC]{MỘT SỐ PHƯƠNG PHÁP PHÂN LOẠI\\ CHO DỮ LIỆU RỜI RẠC}
\subfile{chapters/chapter02}	

\chapter[PHƯƠNG PHÁP PHÂN LOẠI CHO DỮ LIỆU HÌNH ẢNH]{PHƯƠNG PHÁP PHÂN LOẠI \\CHO DỮ LIỆU HÌNH ẢNH}
\subfile{chapters/chapter03}	

\chapter{NHẬN DẠNG ẢNH ỨNG DỤNG TRONG Y HỌC}
\subfile{chapters/chapter04}	

\chapter{KẾT LUẬN VÀ ĐỊNH HƯỚNG NGHIÊN CỨU}
\subfile{chapters/chapter05}	

\chapter{PHỤ LỤC}\label{phuluc}
\subfile{chapters/phu_luc}



\begin{thebibliography}{10}\addcontentsline{toc}{chapter}{TÀI LIỆU THAM KHẢO}
	\bibliographystyle{plain}
	\section*{Tài liệu Tiếng Việt}

\bibitem{NVT} Nguyễn Văn Tuấn (2020). Mô hình hồi quy và khám phá khoa học. NXB Tổng hợp TP. HCM, ISBN: 978-604-58-5625-3, năm 2020.
\bibitem{3} Nghiêm Quan Thường (2016). \textit{Phân loại bằng phương pháp Bayes và các vấn đề liên quan}. Luận văn thạc sĩ. Trường Đại học Cần Thơ.
\bibitem{10} Võ Văn Tài, Phạm  Gia  Thụ, và Tô Anh Dũng, \textit{Sai  số  Bayes  và  khoảng cách  giữa  hai hàm  mật độ xác suất trong phân loại hai tổng thể}, Tạp chí phát  triển khoa  học công nghệ, Đại học Quốc gia TPHCM, 11(6): 23 – 37, 2008. 

	\section*{Tài liệu ngoại văn}	
\bibitem{NVTh} Che-Ngoc, H., Nguyen-Trang, T., Nguyen-Bao, T. et al. A new approach for face detection using the maximum function of probability density functions. Ann Oper Res (2020). https://doi.org/10.1007/s10479-020-03823-1
\bibitem{1} Vo-Van, Tai \& Che Ngoc, Ha \& Nguyen-Trang, Thao. (2017). Textural Features Selection for Image Classification by Bayesian Method. 10.1109/FSKD.2017.8393365.
\bibitem{2} Pham-Gia, T., Turkkan, N. \& Bekker, A. Bounds for the Bayes Error in Classification: A Bayesian Approach Using Discriminant Analysis. Stat. Meth. \& Appl. 16, 7–26 (2007). https://doi.org/10.1007/s10260-006-0012-x
\bibitem{43} Nguyen-Trang, Thao \& Vo-Van, Tai. (2016). A new approach for determining the prior probabilities in the classification problem by Bayesian method. Advances in Data Analysis and Classification. 11. 10.1007/s11634-016-0253-y. 
\bibitem{11} Vovan, T., Tranphuoc, L. \& Chengoc, H. Classifying Two Populations by Bayesian Method and Applications. Commun. Math. Stat. 7, 141–161 (2019). https://doi.org/10.1007/s40304-018-0139-8
\bibitem{32} Vovan, T.. “L1-distance and classification problem by Bayesian method.” Journal of Applied Statistics 44 (2017): 385 - 401.
\bibitem{52} Nguyen-Trang, T., Vo-Van, T. A new approach for determining the prior probabilities in the classification problem by Bayesian method. Adv Data Anal Classif 11, 629–643 (2017). https://doi.org/10.1007/s11634-016-0253-y
\bibitem{Anderson} Anderson J.A. (1982). Logistic discrimination, in P.R. Krishnaiah and L.N. Kanal (Eds.), Classification, Pattern Recognition and Reduction of Dimensionality, Vol. 2 of Handbook of Statistics, 169–191, North Holland, Amsterdam.
\end{thebibliography}


\printindex
\end{document}

% Mẫu luận văn ĐHCT soạn thảo bằng LaTeX, phiên bản 1.0
% ====================================================================
% Tác giả: 				Dương Lữ Điện
% Website:				www.duongludien.com
% GitHub repository:	https://github.com/duongludien/thesis-template
% ====================================================================
